\section{Empirical Framework}

The NBA consists of 30 teams that compete over 82 games. The number of wins at the end of each season determines their ranking, which in turn determines their playoff positioning. 

This study covers all regular seasons played between 1978 and 2018, which includes 1,044 individual seasons, 30 teams, X players, and 164 head coaches. All the data is sourced from the Basketball Reference website. The unit of analysis is the win percentage of a coach-team pairing in a given season. The use of percentages help account for \textit{lockout} seasons where the total number of games played was reduced. Hereafter, a non-interrupted coach-team relationship will be referred to as a \textit{tenure} \footnote{In a limited number of cases, a coach is observed on the same time on separate occasions, typically a few years apart. This includes Flip Saunders (Minnesota Timberworlves), and X()} . To be considered for analysis, tenures are required to last at least one full season. This leaves out head coaches who are hired on interim basis and are unlikely to leave a footprint. Only full-season are considered in the analysis.

The average career length of an NBA head coach is 7.2 seasons, spanning 2.1 teams (Table 1).  Around 60 percent of coaches are observed on more than one team, a much higher ratio than in other sports leagues. \footnote{For example, the German BundesLiga (professional soccer) only has a 30 percent mover ratio.} The average length of a head coach's career spans 7.2 seasons and typically covers 2.1 tenures \footnote{One tenure represents an interrupted team-coach relationship}. These figures are subject to high standard deviations as some coaches spend as little as one season while others, such as Don Nelson, have spent as long as 31 seasons.

In addition to win tallies and coaching profiles, this study uses performance-based metrics as a measure of roster ability. More specifically, a player's relative ability level at the start of a given season is indicated by his overall Box Plus/Minus (BPM) metric from the previous season. BPM is widely recognized as a comprehensive measure of ability and is known for ease-of-interpretation. A player with a BPM of 0 is on-par with league average while a BPM of 5 or more is indicative of all-star status. Historically, only two players have been able to achieve a BPM of 12.5 or more: Michael Jordan and Lebron James. 



I construct a coach-team matched panel dataset and trace the performance of head coaches as they move around the league. I use performance-based metrics to control for player talent and count the number of games missed as a proxy for injuries.
