\documentclass[main.tex]{subfiles}
%\color{blue}
\begin{document}

\begin{abstract}

The degree to which coaching impacts the outcome of a basketball season is much-debated among armchair analysts and anecdotally by those in the sports business, but has been subject to little systematic study. This paper quantifies the managerial contribution of coaches in professional basketball, where front office changes have been frequent enough to disentangle their individual impact. It finds that some coaches, like Greg Popovich, contribute as much as 16 additional wins when roster characteristics are held at their means. The resulting metrics are then regressed on observable characteristics such as playing experience and professional backgrounds, in the aim of identifying secret ingredients for an excellent coach. 

\end{abstract}

\section{Introduction}

Like managers in corporate organizations, coaches play an important role in the outcome for sports franchises. They set strategy, motivate, and establish a culture. Great coaches spark the best performance from individual members and orchestrate the playing style of the collective. Leadership experts concur that in order to be effective, leaders must display emotional intelligence skills that have a direct impact on the operating atmosphere of the team. Basketball coaches are no different. Those who are successful are able to develop close relationships with their players, discover and nurture talent, and adopt unique playing styles. On the court, this translates into exceptional teamwork and seamless collaboration.

The task of evaluating coaches remains tricky. Coaching ability is tied with results and sports leagues are such that there is only one winner. The most successful coaches win trophies but the vast majority doesn't.  Varying roster levels are such that some may have a deeper starting points than others. Still, some coaches manage to stand out. When asked about his decision to join the Boston Celtics, 29 year-old all-star Kemba Walker was quick to mention head coach Brad Stevens. By consistently out-performing expectations, coach Stevens has established himself as a great leader, which has allowed him to attract more talent. Without the rings to show for, however, whether he belongs among the best coaches remains a matter of conjecture.  

Using data from the National Basketball Association, this paper identifies excellent coaches by tracking their performance across teams. The NBA is unique in its high frequency of front office changes, with around 60 percent of its coaches being observed on more than one team. Econometrically, this enhances the accuracy of the estimates and efficiently captures the coaching effect. The resulting estimates put into perspective the contribution of each head coach in NBA history. For example, changing from the mean head coach to Rick Carlisle generates an estimate 10.5 additional wins when roster characteristics are held at their means [Table 1], while changing to Don Nelson generates 3.8 additional wins. 

The analysis spans 1,044 individual seasons played by 30 franchises, and tracks 164 head coaches who featured in the NBA since 1980. In addition to providing a historical ranking, this quantification exercise allows to investigate observable characteristics that shape excellent coaches. For example, the notion that ex-players make for good coaches is debunked in the data. On the contrary, coaches with more diverse backgrounds seem to be better-suited to lead an NBA franchise. Moreover, playoff-seasoned coaches display better results on average. Such findings help identify favorable traits that should be sought after by team executives when considering their next candidate. 

Finally, the coaching estimates prove to be statistically-robust in predicting future performance of theoretical coach-team pairings. 

The impact of managers is well-documented in the literature, both within and outside of sports. According to Bertrand and Schoar, the educational background of CEOs matters for financial outcomes for corporate firms. Those with an MBA degree, for example, adopt more aggressive leadership styles than older cohorts who are more conservative in their decision making. Lazear et Al. show that replacing a "bad" boss can have a greater effect on team output than adding an extra worker. Within the realm of sports, Muelheusseur et Al. find that coaches in professional German soccer have a high variability in contribution. 

The findings presented in this paper should be of value to fans, pundits, and team executives. The structure is as follows. The next section focuses on the data and the advantages of focusing on basketball, followed by a detailed description of the estimation model. The "Results" section lays out the coaching ability estimates, including an analysis of overrated and underrated coaches as well as robustness checks. Section 5 looks into the makings of great coaches, investigating traits that are associated with coaching ability. Conclusions are drawn at the end. 


 \end{document}
 