\documentclass[main.tex]{subfiles}
%\color{blue}
\begin{document}

\section{Introduction}

Does coaching matter? This question is much debated among armchair analysts and anecdotally by those in the sports business, but has been subject to little systematic study. Coaches are subjects to confounding claims: at times they are viewed as critical to team success, while others believe talent is  the sole driver of it. Despite this controversy, organizations continue to pay them enormous amounts. So is their compensation justified? This paper tackles this question through the lens of professional basketball, where coaching changes have been frequent enough to disentangle their individual contributions. It aims to assess the role of coaches by quantifying their relative impact on team success. Not only does this paper cater to the sports community, it also sheds light more broadly on the effect of managers in organizational settings. 

Basketball coaches are similar to corporate managers in many ways. They are key to retaining talent and, no matter how star-studded they are, teams that lack great managers are bound to suffer. The literature finds that managers are capable of altering performance across a variety industries. For instance, Bertrand and Schoar \cite{schoar} show that the identity of CEOs is associated with statistically significant improvements in financial outcomes for corporate firms. Similarly, Graham et Al \cite{graham} find a manager's impact can explain a large part of his salary, and Lazear et Al. \cite{lazear} show that replacing a "bad" boss can have a greater effect on team output than adding an extra worker. Muelheusseur et Al. \cite{muehlheusser} find a high degree of variability in coaches' impact by looking at professional German soccer. For example, some seem to elevate player performance to new heights while others underwhelm given team characteristics. Adding fuel to the fire, a recent article in The Economist \cite{economist} claims that players matter more than coaches in Europe's top flight soccer leagues.

This paper adds to this line of research by addressing managerial contribution in the context of basketball. Besides being one of the world's most popular sports leagues, the National Basketball Association (NBA) features a high turnover of coaches. Additionally, their job is under immense scrutiny: they are dubbed as "genius" when teams over-perform and quick to be scapegoated when things go south. For instance, many people link the resurgence of the Boston Celtics to the appointment of Brad Stevens, while the Toronto Raptors were quick to fire Dwayne Casey after a poor showing in the 2017-18 playoffs, only months after being awarded "Coach of the Year". This paper helps put a finger on the extent of the coaches' impact especially when winning records alone don't tell the full story.

Besides being interesting in its own right, the basketball world provides several advantages for the topic at hand. Not only is the data tremendously granular, but the high turnover contributes a great deal of accuracy to the resulting estimates. This paper offers a novel approach in accounting for talent within a coach fixed effect framework. It adopts performance-based metrics to capture the specificity of each player on the roster, as opposed to encompassing the effect of talent within team fixed effects. Controlling for injuries also proves to be particularly important since not all coaches are dealt the same hand of cards when it comes to player fitness.

The results provide insight into the historical impact of head coaches in the NBA. The paper finds that, all else equal, head coach identity is associated with statistically significant effects on team wins. These effects are large enough to make the difference between a middle-of-the-pack team and a championship contender. However, the high degree of variability points to the importance of identifying the right candidate. For example, switching from Monty Williams, who is at the mean of coaches, to Steve Kerr generates approximately 10.3 additional wins, on average, when roster characteristics are held at their means. This amount would have been enough to push four teams to playoff territory in 2018. This ranks Kerr fourth amongst all-time NBA coaches.

%The paper is structured as follows. The next section describes how the data set is compiled followed by a detailed explanation of the quantitative framework. Results are discussed in detail, followed by a robustness check. Conclusions are drawn at the end.

\end{document}
