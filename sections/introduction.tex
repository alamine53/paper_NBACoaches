\documentclass[main.tex]{subfiles}
%\color{blue}
\begin{document}
\begin{abstract}

The degree to which coaching impacts the outcome of a basketball season is much-debated
among armchair analysts and anecdotally by those in the sports business, but has been subject
to little systematic study. This paper quantifies the managerial contribution of coaches in professional basketball, where front office changes have been frequent enough to disentangle their individual impact. It finds that some coaches, like Greg Popovich, may contribute as much as 16 percent wins when roster characteristics are held at their means. The resulting metrics are regressed on observable characteristics such as playing experience and professional backgrounds, in the aim of identifying secret ingredients for an excellent coach. 

\end{abstract}

\section{Introduction}


Like managers in corporate organizations, coaches play a tremendous role in the outcome for sports franchises. Great coaches spark the best performance from individual members and orchestrate the output of the collective. Leadership experts congrue that in order to be effective, coaches must not only set strategy, but also motivate and establish a culture. Yet, coaching ability is tied with results and sports leagues are such that there is only one winner each season. The most successful coaches win trophies but the vast majority doesn't, including those who may be vastly impactful but have had a deeper starting point. Varying roster levels are among the issues that complicate the question of measuring the contribution of head coaches. 

Scientific research has put to rest any debate as to the value of leadership. Effective leaders, according to leadership pioneer Daniel Goleman, display emotional intelligence skills that have a direct impact on the operating atmosphere of a team. Basketball coaches are no different. Successful coaches develop close relationships with their players, are able to discover talent, and adopt unique playing styles. Their teams display exceptional teamwork and collaboration.

The impact of managers is well-documented in the literature, both within and outside of sports. According to Bertrand and Schoar, the educational background of CEOs matters for financial outcomes for corporate firms. Those with an MBA degree, for example, adopt more aggressive leadership styles than older cohorts who are more conservative in their decision making. Lazear et Al. show that replacing a "bad" boss can have a greater effect on team output than adding an extra worker. Within the realm of sports, Muelheusseur et Al. find that coaches in professional German soccer have a high variability in contribution. 

Using data from the National Basketball Association, this paper identifies excellent coaches by tracking their performance across different teams. It proposes statistically-robust approach to controlling for different roster ability levels and injuries. The NBA is unique in its high frequency of front office changes, with around 60 percent of its coaches being observed on more than one team. Econometrically, this enhances the accuracy of the estimates and efficiently captures the coaching effect.
Not only does this allow to compare coaches from different eras, but it helps investigate observable characteristics that make the best of them stand out. 

The resulting estimates allow for identifying excellent coaches, including those who are over-looked by the lack of accolades. They also help compare the contribution of coaches from different eras. For example, changing from the mean head coach to Rick Carlisle generates an estimate 10.5 additional wins when roster characteristics are held at their means [Table 1], while changing to Don Nelson generates 3.8 additional wins. More importantly, this quantification exercise allows to investigate observable characteristics that shape excellent coaches. For example, the notion that ex-players make for good coaches is debunked in the data. On the contrary, coaches with more diverse backgrounds seem to be better-suited to lead an NBA franchise. Moreover, playoff-seasoned coaches display better results on average.

The analysis spans 1,044 individual seasons played by 30 franchises, and tracks 164 head coaches who featured in the NBA since 1980. In addition to quantifying their relative impact of coaches, it investigates the makings of those that are most impactful. For example, the findings debunk the notion that ex-NBA players make for better coaches. However, coaches with significant playoff experience seem to fare better, on average, than others. Moreover, most good coaches appear to contribute the most during their second season, but show signs of "greatness" as early as their first season. A deeper case-study is warranted but it appears that great coaches are those who are not afraid to adopt unconventional strategies. 

The findings presented in this paper should be of value to fans, pundits, and team executives. The structure is as follows. The next section focuses on the data and the advantages of focusing on basketball, followed by a detailed description of the estimation model. The "Results" section lays out the coaching ability estimates, including an analysis of overrated and underrated coaches as well as robustness checks. Section 5 looks into the makings of great coaches, investigating traits that are associated with coaching ability. Conclusions are drawn at the end. 


 \end{document}
 