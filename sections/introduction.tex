\documentclass[main.tex]{subfiles}
%\color{blue}
\begin{document}
\begin{abstract}

Like managers in corporate organizations, coaches affect the performance of sports teams. Not only do they set strategy, but they lead, motivate, and establish a culture. Some successful coaches win trophies but most don't, making it hard to measure their relative contribution. This paper identifies excellent coaches by quantifying their impact on team success. Using data from the NBA, it disentangles the value of coaching by tracking their performance across different teams and controlling for players' individual talent level. Not only does this allow to compare coaches from different eras, but it helps investigate observable characteristics that make the best of them stand out. 

\end{abstract}

\section{Introduction}
    
The degree to which coaching impacts the outcome of a sports season is much-debated among armchair analysts and anecdotally by those in the sports business, but has been subject to little systematic study. Historically, coaching ability is approximated by the number of championships: Those with most silverware are considered all-time greats. What this approach fails to capture, however, is the extent to which a particular head coach plays a role in their team's success. Superior talent may provide some coaches a head start while others may over-achieve given a set of players. 

The impact of managers is well-documented in the literature, both within and outside of sports. According to Bertrand and Schoar, the identity of CEOs as well their educational background and birth cohort determines financial outcomes for corporate firms. Those with an MBA degree, for example, adopt more aggressive leadership styles than older cohorts who are more conservative in their decision making. \footnote{In the context of the paper, MBA-holders are measured against older CEOs who tend to be more conservative in their decision making}.  Lazear et Al. show that replacing a "bad" boss can have a greater effect on team output than adding an extra worker. Within the realm of sports, Muelheusseur et Al. find that coaches in professional German soccer have a high variability in contribution. 

The sports world continues to treat coaches as a "black box" despite some of them having a tremendous impact. In addition to setting strategy, coaches motivate, inspire, and establish a culture. Identifying the right head coach is crucial to maximizing on a team's potential and accentuating its players' relative strengths. The sports world has seen many memorable figures, including the likes Pep Guardiola, Bill Belichik, Phil Jackson, and Greg Popovich. Motivated by their success, this paper views coaching as source of value rather than just a seat to fill. 

This paper proposes a new, statistically-robust approach to measuring the relative contribution of head coaches in professional sports. Using data from the National Basketball Association, it disentangles their impact as they move across teams while controlling for roster quality and injuries. The NBA is unique in its high frequency of front office changes, with around 60 percent of its coaches being observed on more than one team. Econometrically, this enhances the accuracy of the estimates and efficiently captures the coaching effect.

The resulting estimates allow for identifying excellent coaches, including those who are over-looked by the lack of accolades. They also help compare the contribution of coaches from different eras. For example, changing from the mean head coach to Rick Carlisle generates an estimate 10.5 additional wins when roster characteristics are held at their means [Table 1], while changing to Don Nelson generates 3.8 additional wins. More importantly, this quantification exercise allows to investigate observable characteristics that shape excellent coaches. For example, the notion that ex-players make for good coaches is debunked in the data. On the contrary, coaches with more diverse backgrounds seem to be better-suited to lead an NBA franchise. Moreover, playoff-seasoned coaches display better results on average.



This paper proposes a new, statistically-robust approach to measuring the impact of coaches. Using data from the National Basketball Association, it exploits the high rate of front office changes to track the performance of coaches as they move across different teams. Econometrically, it solves for the problem of having different starting points by capturing the effect of roster composition. This allows to identify coaches who are over-looked by mediocre success records and helps put into perspective the relative impact of coaches from different eras. The data covers all 164 head coaches who featured in the NBA since 1980 and spans 1,044 individual seasons played by 30 franchises. 


In addition to quantifying the relative impact of coaches and identifying excellent ones, this paper investigates the makings of those that appear at the top of the distribution scale. First and foremost, the findings debunk the notion that ex-NBA players make for better coaches. However, coaches with significant playoff experience seem to fare better, on average, than others. Moreover, most good coaches appear to contribute the most during their second season, but show signs of "greatness" as early as their first season. A deeper case-study is warranted but it appears that great coaches are those who are not afraid to adopt unconventional strategies. 

The findings presented in this paper should be of value to fans, pundits, and team executives. The structure is as follows. The next section focuses on the data and the advantages of focusing on basketball, followed by a detailed description of the estimation model. The "Results" section lays out the coaching ability estimates, including an analysis of overrated and underrated coaches as well as robustness checks. Section 5 looks into the makings of great coaches, investigating traits that are associated with coaching ability. Conclusions are drawn at the end. 

\section{Data}

The NBA consists of 30 teams that compete over 82 games. The number of wins at the end of each season determines their ranking. \footnote{The post-season extends until June but this study is limited to the regular season} Using data between 1978 and 2017, I construct a coach-team matched panel dataset and trace the performance of head coaches as they move around the league. I use performance-based metrics to control for player talent and count the number of games missed as a proxy for injuries.

Coaching changes are more frequent in the NBA than in other sports leagues. On average, teams replace their head coach every X years and each head coach averages 2.1 teams over the course of his career. Most coaching changes occur prior to the start of a season but some may take place mid-season. For the sake of this analysis, only full-season are considered in the analysis. In other words, all mid-season hires are considered to take place in the following off-season. 

The total number of NBA head coaches covered in this study is 164. This includes all of those who appeared between the 1977-78 and the 2016-17 season. To be considered for analysis, coaches are required to spend at least one full season on a given team. This condition leaves out those hired on interim basis because they are unlikely to leave a footprint. Out of the initial sample of 164 head coaches, 154 remain after after this condition is applied. Table 1 provides summary statistics for all the coaches in the sample. Around 60 percent of coaches are observed on more than one team, a much higher ratio than in other sports leagues. \footnote{For example, the German BundesLiga (professional soccer) only has a 30 percent mover ratio.} The average length of a head coach's career spans 7.2 seasons and typically covers 2.1 tenures \footnote{One tenure represents an interrupted team-coach relationship}. These figures are subject to high standard deviations as some coaches spend as little as one season while others, such as Don Nelson, have spent as long as 31 seasons.

\section{Method}

The main estimation strategy is a coach-fixed effect (FE) regression model with team wins as a dependent variable. As control variables, the model includes performance-based metrics that help capture a player's relative skill level. More specifically, it is assumed that a player's Box Plus Minus (BPM) metric from his last season is an overall measure of his ability level. BPM is a 

The coaching impact is captured by the fixed effect ($\lambda_{k}$), which is a dummy variable designating head coach identity. The resulting fixed effects are algebraically equivalent to deviations from means, so they would represent the additional wins relative to the average. The model also includes player metrics as a control for player ability, and the number of games missed as a proxy for player fitness. 

The model uses Box Plus Minues (BPM) as an overall measure of player ability. More specifically, a player expected level of ability is given by his BPM from the previous season. 

Formally, the model can be represented by the following equation: 

For team $i$ in season $t$ under coach $k$,

\begin{equation}
\text{Wins}_{itk} = \sum_{p=1}^{5}(\alpha_{p}\text{BPM}_{p,t-1} + \beta_{p}\text{Inj}_{pit}) + \lambda_{k} + \epsilon_{itk}
\end{equation}

where ${BPM}_{p,t-1}$ represents player $p$'s achieved BPM score in season $t-1$, ${Inj}_{p,i,t}$ indicates his number of games missed (presumably due to injury), and $\lambda_{k}$ is the coach fixed effect. $\epsilon_{i,t,k}$ is the error term. 

For simplicity, the model assumes a 5-player roster per team. Since players are sorted by minutes played, where Player 1 ($p = 1$) would be the coach's preferred player by the defining characteristic of seeing the most playing time. Player 2 would be the second and Player 5 would be the least in a 5-person rotation. It is important to note that reducing teams to 5 members does not compromise accuracy. Although rosters may extend up to 12 members, coaches mostly rely on a handful of players. Secondary players see significant variation in roles and personnel over the course of a season, so they are excluded from the main estimation. Section 5 considers larger roster sizes and confirms that the results are robust to the inclusion of additional players.

There is enough reason to believe that neither time or team fixed effects play a significant role because no franchise offers strict advantages over another. Although one might argue that big market teams like the Lakers in Los Angeles and the Knicks in New York may have a slight advantage in attracting talent, expectations are also high for those teams and they often experience periods of downturn.  In addition, the institutional setup of the NBA is such that no team has an edge when it comes to offering higher salaries. Unlike European soccer and other professional sports leagues, the NBA adopts a salary cap where teams have a limit on how much they can spend. Moreover, the structure of the league also compensates "losing" teams by granting them priority in selecting players out of college. All of these factors ensure that the league is somewhat balanced out. 

can offer more money than others to attract a player. because of a common salary cap. \footnote{There are exceptions like for retaining a player of their own.} In addition, the lottery system ensures that weaker teams have the chance to rebuild through the draft. 

\subsection{Solving for Serial Correlation}

The main challenge arises from the use of performance-based metrics as indicators for player ability, which could pick up some of the coaching effect and lead to potentially understating the impact of coaches over the long run. In order to improve unbiasedness, it imposes a cap on the time span over which the coach effect is examined. In other words, only the first \textit{three} seasons from each coach-team tenure are observed, filtering out the rest. For example, Phil Jackson's 9-season tenure with the Chicago Bulls (1990 to 1998) is reduced to the first \textit{three} (1990 to 1992) by eliminating all subsequent seasons. This strategy helps the playing field by assessing all coaches over a fixed time span. The downside is a reduction in sample size, which drops from X to Y observations.

Trimming-off seasons beyond the third has effectively little impact on the accuracy of the estimates. A 3-season tenure is as almost equally informative as a 6-season tenure when it comes to the coach effect. Section 5 helps illustrate this point by showing that results are robust to different time considerations. For simplicity, the main estimation model assumes a 3-season cap because it allows ample time for coaches to leave a footprint without diving into the danger zone of serial correlation.

Somel papers choose to use payroll as a proxy for ability, as opposed to performance-based metrics. However, salaries fail to pick up a great deal of variations, so this paper avoids this alternative. For instance, players may witness a jump in performance from one season to another, yet salaries are sticky over the course of a multi-year contract. Additionally, the NBA is under a salary cap system where pay is not always proportional to on-court value. Many argue that, because of the "maximum" contract, the relative impact of a player like Lebron James outweighs the amount of salary-based income he receives. Moreover, star players frequently accept pay cuts to team up with other stars. All of these points make payroll a weak substitute for ability. 

\section{Results}

\subsection{Joint Impact of Coaches}

In aggregate, coach fixed effects explain a significant portion of the variation in team wins. Table 3 displays the results from two models, a baseline Model (I) that doesn't include fixed effects and Model (II) that includes them. A comparison between (I) and (II) highlights the importance of including them. For example, the adjusted R-squared increases by 12.8 percentage points, pointing to greater explanatory power in Model (II). Additionally, the coefficients on Players 1 through 5 experience a decrease in magnitude in (II). This decrease is highest in absolute terms for players 1 and 5, which indicates that much of these players' performance is associated with the roles assigned to them by their coach. In other words, great coaches tend to make a better use of their best and worst player in the main rotation. Moreover, the F-test on the joint significance of coaches is highly significant, underlining the importance of including them. 

%\subfile{tables/table3.tex}

By looking at Model (II), we can infer that roster quality and injuries are important determinants of success, with statistically significant coefficients across the board. On average, a full point increase in Player 1's BPM is associated with 1.73 additional wins for team \textit{i}, compared to 0.84 for player 5. This suggests that, on average, player 1 is almost twice as important for success as player 5. In terms of injuries, a game missed by player 1 is associated with twice as many losses as player 5. For example, a player like Stephen Curry who achieves a BPM of 12 is associated with around 18 team wins and every game he misses is associates lowers wins by 0.17. These results confirm general intuition that star players matter to a great degree.

\subsection{Individual Coach Fixed Effects}

In terms of individual fixed effects, some coaches seem to make a bigger difference than others. Table 4 presents the estimations corresponding to each of the 164 head coaches represented in the sample, each one assigned a single coefficient. \footnote{Coefficients for \textit{non-movers} are greyed-out because they are subject to lower accuracy.} The coefficients are significant in magnitude and are subject to high variability. This means that not only do coaches improve performance by a statistically significant margin, but the choice of head coach matters to a large extennt. 

The size of coefficients is relative to a reference category, which means that the magnitude and significance levels are subject to vary depending on the choice of reference category. In this case, the reference category is chosen as the mean head coach. 

The high degree of heterogeneity in the resulting estimates indicates that the implications for making the \textit{right} hire are substantial. Given the magnitude of the coefficients, many head coaches are able to propel teams into playoff territory or make them championship contenders. Although my estimates are based solely on 3-seasons worth' of regular season performance, 7 of the top 10 coaches happen to be NBA champions. This suggests that organizations who hire the right managers are more likely to reach the ultimate prize. 

\subsubsection{Exceptional Coaches}

By descending order of impact, the top coaches in the resulting ability estimates include Gregg Popovich (13.0), Brad Stevens (12.3), Tom Thibodeau (12.3), Pat Riley (10.5), Larry Bird (10.4), Steve Kerr (10.3), Rick Carlisle (9.6), Dave Joerger (9.5), and Billy Cunningham (9.1). As such, switching from Monty Williams who is at the mean of coaches,  to Gregg Popovich generates an additional 13.0 wins when roster characteristics are held at their means. To put this into perspective, this would have been enough to push Williams’ Pelicans to a playoff position in 2017.

\subsubsection{Overrated Coaches}

To get a feel for these estimates, it is helpful to compare between the left-hand side and the right-hand side of Table 4, where estimated fixed effects are compared against standard win percentages. Coaches who figure at the top of the ladder on the right-hand side but not the left can be considered as \textit{overrated} in the sense that their success has had a lot to do with the quality of the roster they've inherited. A prime example is Mike Brown who figures at 11 on the right but 67 on the left. In other words, at the time of his tenure, the roster's talent level [see LeBron James] helped propel his team to higher winning percentages. Other similar instances include David Blatt,....

\subsubsection{Underrated Coaches}

Similarly, this comparative exercise can help identify coaches who have had a great impact despite their team's mediocre end-of-season record. A primary case-in-point is Brad Stevens whose average wins over the course of his first three seasons ranks him 69th but whose estimated fixed effect ranks him second overall. In such cases, it can be said that teams tend to exceed expectations under this head coach. In Stevens' case, his team has achieved an average 38 wins with a roster equivalent to around 12.3 wins less than that. 

 \end{document}
 