
\section{Method}

The main estimation strategy is a coach-fixed effect (FE) regression model with team wins as a dependent variable. As control variables, the model includes performance-based metrics that help capture a player's relative skill level. More specifically, it is assumed that a player's Box Plus Minus (BPM) metric from his last season is an overall measure of his ability level. BPM is a 

The coaching impact is captured by the fixed effect ($\lambda_{k}$), which is a dummy variable designating head coach identity. The resulting fixed effects are algebraically equivalent to deviations from means, so they would represent the additional wins relative to the average. The model also includes player metrics as a control for player ability, and the number of games missed as a proxy for player fitness. 

The model uses Box Plus Minues (BPM) as an overall measure of player ability. More specifically, a player expected level of ability is given by his BPM from the previous season. 

Formally, the model can be represented by the following equation: 

For team $i$ in season $t$ under coach $k$,

\begin{equation}
\text{Wins}_{itk} = \sum_{p=1}^{5}(\alpha_{p}\text{BPM}_{p,t-1} + \beta_{p}\text{Inj}_{pit}) + \lambda_{k} + \epsilon_{itk}
\end{equation}

where ${BPM}_{p,t-1}$ represents player $p$'s achieved BPM score in season $t-1$, ${Inj}_{p,i,t}$ indicates his number of games missed (presumably due to injury), and $\lambda_{k}$ is the coach fixed effect. $\epsilon_{i,t,k}$ is the error term. 

For simplicity, the model assumes a 5-player roster per team. Since players are sorted by minutes played, where Player 1 ($p = 1$) would be the coach's preferred player by the defining characteristic of seeing the most playing time. Player 2 would be the second and Player 5 would be the least in a 5-person rotation. It is important to note that reducing teams to 5 members does not compromise accuracy. Although rosters may extend up to 12 members, coaches mostly rely on a handful of players. Secondary players see significant variation in roles and personnel over the course of a season, so they are excluded from the main estimation. Section 5 considers larger roster sizes and confirms that the results are robust to the inclusion of additional players.

There is enough reason to believe that neither time or team fixed effects play a significant role because no franchise offers strict advantages over another. Although one might argue that big market teams like the Lakers in Los Angeles and the Knicks in New York may have a slight advantage in attracting talent, expectations are also high for those teams and they often experience periods of downturn.  In addition, the institutional setup of the NBA is such that no team has an edge when it comes to offering higher salaries. Unlike European soccer and other professional sports leagues, the NBA adopts a salary cap where teams have a limit on how much they can spend. Moreover, the structure of the league also compensates "losing" teams by granting them priority in selecting players out of college. All of these factors ensure that the league is somewhat balanced out. 

can offer more money than others to attract a player. because of a common salary cap. \footnote{There are exceptions like for retaining a player of their own.} In addition, the lottery system ensures that weaker teams have the chance to rebuild through the draft. 

\subsection{Solving for Serial Correlation}

The main challenge arises from the use of performance-based metrics as indicators for player ability, which could pick up some of the coaching effect and lead to potentially understating the impact of coaches over the long run. In order to improve unbiasedness, it imposes a cap on the time span over which the coach effect is examined. In other words, only the first \textit{three} seasons from each coach-team tenure are observed, filtering out the rest. For example, Phil Jackson's 9-season tenure with the Chicago Bulls (1990 to 1998) is reduced to the first \textit{three} (1990 to 1992) by eliminating all subsequent seasons. This strategy helps the playing field by assessing all coaches over a fixed time span. The downside is a reduction in sample size, which drops from X to Y observations.

Trimming-off seasons beyond the third has effectively little impact on the accuracy of the estimates. A 3-season tenure is as almost equally informative as a 6-season tenure when it comes to the coach effect. Section 5 helps illustrate this point by showing that results are robust to different time considerations. For simplicity, the main estimation model assumes a 3-season cap because it allows ample time for coaches to leave a footprint without diving into the danger zone of serial correlation.

Somel papers choose to use payroll as a proxy for ability, as opposed to performance-based metrics. However, salaries fail to pick up a great deal of variations, so this paper avoids this alternative. For instance, players may witness a jump in performance from one season to another, yet salaries are sticky over the course of a multi-year contract. Additionally, the NBA is under a salary cap system where pay is not always proportional to on-court value. Many argue that, because of the "maximum" contract, the relative impact of a player like Lebron James outweighs the amount of salary-based income he receives. Moreover, star players frequently accept pay cuts to team up with other stars. All of these points make payroll a weak substitute for ability. 